\documentclass[docid=2019]{comp_exam_round2}
\begin{document}
\setcounter{chapter}{2018}
\exam{Exam 2019}

\examgroup{}

Considering the G1 grammar below (parentheses and numbers on the right identify the productions):

\begin{center}
    \begin{minipage}{0.15\textwidth}
        Grammar G1:

        \begin{enumerate}[wide, noitemsep]
            \setcounter{enumi}{1}
            \item \texttt{S → w}
            \item \texttt{S → a C b}
            \item \texttt{C → $\varepsilon$}
            \item \texttt{C → S x C}
        \end{enumerate}
    \end{minipage}
\end{center}

\question
Indicate the First and Follow sets for the grammar variables;

\ansseparator

\vspace{-2.0em}
\begin{alignat*}{5}
    & First(S) &&= \{ w, a              \} && ~~~~~~~~~~ && Follow(S) &&= \{ \$, x \} \\
    & First(C) &&= \{ \varepsilon, w, a \} && ~~~~~~~~~~ && Follow(C) &&= \{ b     \}
\end{alignat*}
\vspace{-3.0em}

\question
Show the LL(1) parsing table. Is the grammar LL(1)?

\ansseparator

\begin{center}
    \small
    \begin{tabular}{@{} c | p{30mm} | p{30mm} | p{30mm} | p{30mm} @{}}
        \multirow{2}{*}{NT} & \multicolumn{3}{c}{T} \\ \cline{2-5}
            & $w$ & $a$ & $b$ & $x$ \\ \hline
        $S$ & \texttt{S → w}     & \texttt{S → a C b} & \\ \hline
        $C$ & \texttt{C → S x C} & \texttt{C → S x C} & \texttt{C → $\varepsilon$}
    \end{tabular}
\end{center}

The grammar is LL(1) because no cell in the parsing table has more than one production.

\question
Is the grammar ambiguous? Justify your answer;

\ansseparator

\noindent
The grammar is unambiguous, because that follows directly from the fact it is a LL(k) grammar with finite $k=1$.

\question
Determine and show the LR(0) automaton (consider that for the LR parser, S' is the new start variable and the production 1. \texttt{S' → S \$} is added to the grammar);

\ansseparator

\begin{minipage}[t]{0.49\textwidth}
The grammar is:
\begin{enumerate}[wide, noitemsep]
    \item \texttt{S' → S \$}
    \item \texttt{S → w}
    \item \texttt{S → a C b}
    \item \texttt{C → $\varepsilon$}
    \item \texttt{C → S x C}
\end{enumerate}
\end{minipage}
\begin{minipage}[t]{0.49\textwidth}
The list of possible items is:
\begin{itemize}[wide, noitemsep]
    \item \texttt{S' → $\cdot$S\$}
    \item \texttt{S' → S$\cdot$\$}
    \item \texttt{S → $\cdot$w}
    \item \texttt{S → w$\cdot$}
    \item \texttt{C → $\varepsilon\cdot$}
    \item \texttt{C → $\cdot$SxC}
    \item \texttt{C → S$\cdot$xC}
    \item \texttt{C → Sx$\cdot$C}
    \item \texttt{C → SxC$\cdot$}
\end{itemize}
\end{minipage}

\vspace{-1em}
\begin{center}
    \begin{tikzpicture}[->,>=stealth',node distance=3cm,initial text=$ $,]
        \node[draw, align=left                   ] (s0) {
            $s_0$:\\
            \texttt{S' → $\cdot$S\$} \\
            \texttt{S~ → $\cdot$w} \\
            \texttt{S~ → $\cdot$aCb}
        };
        \node[draw, align=left, right of=s0      ] (s1) {
            $s_1$:\\
            \texttt{S' → S$\cdot$\$}
        };
        \node[draw, align=left, below of=s1      ] (s2) {
            $s_2$:\\
            \texttt{S → w$\cdot$}
        };
        \node[draw, align=left, below of=s0      ] (s3) {
            $s_3$:\\
            \texttt{S → a$\cdot$Cb} \\
            \texttt{S → $\cdot$w} \\
            \texttt{S → $\cdot$aCb} \\
            \texttt{C → $\varepsilon\cdot$} \\
            \texttt{C → $\cdot$SxC}
        };
        \node[draw, align=left, below of=s3      ] (s6) {
            $s_6$:\\
            \texttt{S → aCb$\cdot$}
        };
        \node[draw, align=left, left of=s6      ] (s4) {
            $s_4$:\\
            \texttt{S → aC$\cdot$b}
        };
        \node[draw, align=left, below of=s4      ] (s5) {
            $s_5$:\\
            \texttt{S → aCb$\cdot$}
        };
        \node[draw, align=left, right of=s6      ] (s7) {
            $s_7$:\\
            \texttt{S → $\cdot$aCb}\\
            \texttt{S → $\cdot$w}\\
            \texttt{C → Sx$\cdot$C}\\
            \texttt{C → $\varepsilon\cdot$}\\
            \texttt{C → $\cdot$SxC}
        };
        \node[draw, align=left, right of=s7      ] (s8) {
            $s_8$:\\
            \texttt{C → SxC$\cdot$}
        };
        

        \draw
                (s0)	edge[above] node{S} (s1)
                (s0)	edge[above] node{w} (s2)
                (s0)	edge[left ] node{a} (s3)

                (s3)	edge[loop left, looseness=5] node{a} (s3)
                (s3)	edge[above] node{w} (s2)
                (s3)	edge[above] node{c} (s4)
                (s3)	edge[left] node{S} (s6)
                
                (s4)	edge[left] node{b} (s5)

                (s6)	edge[above, bend left=8] node{x} (s7)
                
                (s7)	edge[above] node{a} (s3)
                (s7)	edge[below, bend left=8] node{S} (s6)
                (s7)	edge[right] node{w} (s2)
                
                (s7)	edge[above] node{C} (s8)
                
                ;
    \end{tikzpicture}
\end{center}

\question
Is the grammar LR(0)? Justify your answer indicating the LR(0) parsing table corresponding to the automaton;

\ansseparator

\begin{center}
    \begin{tabular}{c || l | l | l | l | l | l | l}
              & a & b & w & x & \$ & S & C \\ \hline\hline
        $s_0$ & shift $s_3$ &   & shift $s_2$ &   &    & goto $s_1$ &   \\ \hline
        $s_1$ & & & & & accept & \\ \hline
        $s_2$ & reduce 2 & reduce 2 & reduce 2 & reduce 2 & reduce 2 &   &   \\ \hline
        $s_3$ & \begin{tabular}{@{}l@{}}shift $s_3$\\reduce 4\end{tabular} & reduce 4 & \begin{tabular}{@{}l@{}}shift $s_2$\\reduce 4\end{tabular} & reduce 4 & reduce 4 & goto $s_6$ & goto $s_4$ \\ \hline
        $s_4$ &   & shift $s_5$ &   &   &    &   &   \\ \hline
        $s_5$ & reduce 3 & reduce 3 & reduce 3 & reduce 3 & reduce 3  &   &   \\ \hline
        $s_6$ &   &   &   & shift $s_7$ &  &   &   \\ \hline
        $s_7$ & \begin{tabular}{@{}l@{}}shift $s_3$\\reduce 4\end{tabular} & reduce 4 & \begin{tabular}{@{}l@{}}shift $s_2$\\reduce 4\end{tabular} & reduce 4 & reduce 4 & shift $s_6$ & shift $s_8$ \\ \hline
        $s_8$ & reduce 5 & reduce 5 & reduce 5 & reduce 5 & reduce 5 &   &  
    \end{tabular}
\end{center}

The grammar is not LR(0), because it has four reduce/shift conflicts in its parsing table, mainly caused by production \texttt{C → $\varepsilon$}.

\question
Explain in which circumstances a value of k > 1 can make a non LL(1) CFG into an LL(k) CFG.
If needed, give examples to support your explanations.

\ansseparator

If a grammar is strictly LL(k) ($k>1$) then it is not LL(i), $i < k$, meaning it is not LL(1). For instance, the grammar

\begin{enumerate}[wide, noitemsep]
    \item \texttt{S → ab}
    \item \texttt{S → ac}
\end{enumerate}

\noindent
is not LL(1), because analysing the first token is not enough to disambiguate it; however, if we read the second token we can then decide which production to use (if we see \texttt{b}, we apply \texttt{S → ab}; if we see \texttt{c}, we apply \texttt{S → ac}).

\examgroup{}

TODO

\examgroup{}

\question
Comment the following sentence: ``The
only way to achieve a high-level representation
with trees representing expressions according
to the rules of precedence of the operations of
the input language is to have the grammar of
the language taking care of the precedence
rules.''

\ansseparator

\noindent
False. One can obtain the high-level representation with trees by obtaining the AST, and the AST can be directly obtained if the parser makes certain provisions/corrections to prevent some nodes from being added to the tree.

\examgroup{}

TODO

\examgroup{}

TODO

\end{document}
