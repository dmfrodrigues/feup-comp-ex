\documentclass[docid=2019]{comp_exam}
\begin{document}
\setcounter{chapter}{2018}
\exam{Exam 2019}
{

% \examgroup{Group I}
% \questionitem{Item a}
% \begin{minipage}[c]{0.50\textwidth}
% 	The methodology should follow these lines:
% 	\begin{enumerate}
% 		\item Design an automaton that recognises strings of even length.
% 		\item Design an automaton that recognizes strings ending in 11.
% 		\item Join the two automata using a single initial state with $\varepsilon$-transitions to each of the automata
% 	\end{enumerate}
% 	\begin{center}
% 		\begin{tabular}{r | c c c}
% 			$\delta_N$           & 0            & 1            & $\varepsilon$ \\ \hline
% 			$\rightarrow q_0   $ & $\emptyset $ & $\emptyset $ & $\{q_{00}, q_{10}\}$ \\
% 			$         ^* q_{00}$ & $\{q_{01}\}$ & $\{q_{01}\}$ & $\emptyset$ \\
% 			$            q_{01}$ & $\{q_{00}\}$ & $\{q_{00}\}$ & $\emptyset$ \\
% 			$            q_{10}$ & $\{q_{10}\}$ & $\{q_{11}\}$ & $\emptyset$ \\
% 			$            q_{11}$ & $\{q_{10}\}$ & $\{q_{12}\}$ & $\emptyset$ \\
% 			$         ^* q_{12}$ & $\{q_{10}\}$ & $\{q_{12}\}$ & $\emptyset$
% 		\end{tabular}
% 	\end{center}
% \end{minipage}
% \begin{minipage}[c]{0.49\textwidth}
% 	\begin{center}
% 		\begin{tikzpicture}[->,>=stealth',node distance=2.3cm,initial text=$ $,]
% 			\node[state, initial			] (q0) {$q_0$};
% 			\node[state, accepting,above right of=q0	] (q00) {$q_{00}$};
% 			\node[state, right of=q00		] (q01) {$q_{01}$};
% 			\node[state, below right of=q0	] (q10) {$q_{10}$};
% 			\node[state, right of=q10		] (q11) {$q_{11}$};
% 			\node[state, accepting,right of=q11		] (q12) {$q_{12}$};

% 			\draw	(q0)	edge[left				] node{$\varepsilon$} (q00)
% 					(q0)	edge[left				] node{$\varepsilon$} (q10)

% 					(q00)	edge[bend right,below	] node{$0,1$} (q01)
% 					(q01)	edge[bend right,above	] node{$0,1$} (q00)

% 					(q10)	edge[loop below			] node{$0$} (q10)
% 					(q10)	edge[above				] node{$1$} (q11)

% 					(q11)	edge[above				] node{$1$} (q12)
% 					(q11)	edge[bend left,below	] node{$1$} (q10)

% 					(q12)	edge[loop below			] node{$1$} (q12)
% 					(q12)	edge[bend right,above	] node{$1$} (q10)
% 					;
% 		\end{tikzpicture}
% 	\end{center}
% \end{minipage}
% \questionitem{Item b}
% \begin{minipage}[c]{0.49\textwidth}
% 	\begin{alignat*}{2}
% 		&\varepsilon close(q_0    && )=\{q_0,q_{00},q_{10}\}\\
% 		&\varepsilon close(q_{00} && )=\{q_{00}\}\\
% 		&\varepsilon close(q_{01} && )=\{q_{01}\}\\
% 		&\varepsilon close(q_{10} && )=\{q_{10}\}\\
% 		&\varepsilon close(q_{11} && )=\{q_{11}\}\\
% 		&\varepsilon close(q_{12} && )=\{q_{12}\}\\
% 	\end{alignat*}
% \end{minipage}
% \begin{minipage}[c]{0.49\textwidth}
% 	\begin{center}
% 		\begin{tabular}{r | c c}
% 			$\delta_D$                              & 0            & 1           \\ \hline
% 			$\rightarrow ^* \{q_0,q_{00},q_{10}\} $ & $\{q_{01},q_{10}\}$ & $\{q_{01},q_{11}\}$ \\
% 			$            ^* \{q_{00},q_{10}     \}$ & $\{q_{01},q_{10}\}$ & $\{q_{01},q_{11}\}$ \\
% 			$            ^* \{q_{00},q_{11}     \}$ & $\{q_{01},q_{10}\}$ & $\{q_{01},q_{12}\}$ \\
% 			$            ^* \{q_{00},q_{12}     \}$ & $\{q_{01},q_{10}\}$ & $\{q_{01},q_{12}\}$ \\
% 			$               \{q_{01},q_{10}     \}$ & $\{q_{00},q_{10}\}$ & $\{q_{00},q_{11}\}$ \\
% 			$               \{q_{01},q_{11}     \}$ & $\{q_{00},q_{10}\}$ & $\{q_{00},q_{12}\}$ \\
% 			$            ^* \{q_{01},q_{12}     \}$ & $\{q_{00},q_{10}\}$ & $\{q_{00},q_{12}\}$
% 		\end{tabular}
% 	\end{center}
% \end{minipage}
% \begin{center}
% 	\begin{tikzpicture}[->,>=stealth',node distance=3cm,initial text=$ $,]
% 		\node[state, accepting,initial,align=center				] (q0_q00_q10) {$q_0$\\$q_{00}$\\$q_{10}$};
% 		\node[state, right of=q0_q00_q10,align=center			] (q01_q11) {$q_{01}$\\$q_{11}$};
% 		\node[state, above of=q01_q11,align=center				] (q01_q10) {$q_{01}$\\$q_{10}$};
% 		\node[state, accepting,below of=q01_q11,align=center	] (q01_q12) {$q_{01}$\\$q_{12}$};
% 		\node[state, accepting,right of=q01_q10,align=center	] (q00_q10) {$q_{00}$\\$q_{10}$};
% 		\node[state, accepting,right of=q01_q11,align=center	] (q00_q11) {$q_{00}$\\$q_{11}$};
% 		\node[state, accepting,right of=q01_q12,align=center	] (q00_q12) {$q_{00}$\\$q_{12}$};

% 		\draw	(q0_q00_q10)	edge[left		] node{$0$} (q01_q10)
% 				(q0_q00_q10)	edge[above		] node{$1$} (q01_q11)

% 				(q00_q10)		edge[bend left=10,below] node{$0$} (q01_q10)
% 				(q00_q10)		edge[bend left=10] node{$1$} (q01_q11)

% 				(q00_q11)		edge[bend left=10] node{$0$} (q01_q10)
% 				(q00_q11)		edge[bend left=10] node{$1$} (q01_q12)

% 				(q00_q12)		edge[right		 ] node{$0$} (q01_q10)
% 				(q00_q12)		edge[bend left=10,below] node{$1$} (q01_q12)

% 				(q01_q10)		edge[bend left=10,above] node{$0$} (q00_q10)
% 				(q01_q10)		edge[bend left=10] node{$1$} (q00_q11)

% 				(q01_q11)		edge[bend left=10] node{$0$} (q00_q10)
% 				(q01_q11)		edge[bend right=10] node{$1$} (q00_q12)

% 				(q01_q12)		edge[left		 ] node{$0$} (q00_q10)
% 				(q01_q12)		edge[bend left=10,above] node{$1$} (q00_q12)
% 		 		;
% 	\end{tikzpicture}
% \end{center}
% \pagebreak
% \questionitem{Item c}
% From the DFA for $L_1$, $D_1$, we will find an NFA for $L_2$, $N_2$, by considering the same states and following these rules:
% \begin{itemize}
% 	\item The initial state of $N_2$ is a state with a name different from that of any state of $D_1$; say it's $i$.
% 	\item If $q \in Q_{D_1}$ is an initial state, $q \in Q_{N_2}$ is a final state.
% 	\item If $q \in Q_{D_1}$ is a final state, then $q \in \delta_{N_2}(i,\varepsilon)$ (there exists an $\varepsilon$-transition from $r$ to $q$).
% 	\item If $\delta_{D_1}(q,a)=r$, then $q \in \delta_{N_2}(r,a)$ (swap direction of transitions).
% \end{itemize}
% Finally, convert $N_2$ to a DFA.\par
% This technique allows to easily create a DFA that accepts the reverse of all strings for which we already have a DFA.
% \questionitem{Item d}
% \begin{center}
% 	\begin{tabular}{c c c}
% 		\begin{tikzpicture}[->,>=stealth',node distance=2.5cm,initial text=$ $,]
% 			\node[state, accepting,initial	] (1) {$1$};
% 			\node[state, accepting, below right of=1	] (3) {$3$};
% 			\node[state, above right of=3	] (2) {$2$};
% 			\draw	(1)	edge[above		] node{$a,b$} (2)
% 					(2)	edge[loop above	] node{$a$} (2)
% 					(2)	edge[bend right, left	] node{$b$} (3)
% 					(3)	edge[bend right, right	] node{$b$} (2)
% 					(3)	edge[left				] node{$a$} (1);
% 		\end{tikzpicture} &
% 		\begin{tikzpicture}[->,>=stealth',node distance=2.5cm,initial text=$ $,]
% 			\node[state, accepting,initial	] (1) {$1$};
% 			\node[state, accepting, below right of=1	] (3) {$3$};
% 			\node[state, above right of=3	] (2) {$2$};
% 			\draw	(1)	edge[above		] node{$a+b$} (2)
% 					(2)	edge[loop above	] node{$a$} (2)
% 					(2)	edge[bend right, left	] node{$b$} (3)
% 					(3)	edge[bend right, right	] node{$b$} (2)
% 					(3)	edge[left				] node{$a$} (1);
% 		\end{tikzpicture} &
% 		\begin{tikzpicture}[->,>=stealth',node distance=2.5cm,initial text=$ $,]
% 			\node[state, accepting,initial	] (1) {$1$};
% 			\node[state, accepting, right of=1	] (3) {$3$};
% 			\draw	(1)	edge[bend right, below	] node{$(a+b)a^*b$} (3)
% 					(3)	edge[bend right, above	] node{$a$} (1)
% 					(3) edge[loop above			] node{$ba^*b$} (3);
% 		\end{tikzpicture}
% 	\end{tabular}
% \end{center}
% \begin{center}
% 	\begin{tabular}{c | c}
% 		\begin{tikzpicture}[->,>=stealth',node distance=2.5cm,initial text=$ $,]
% 			\node[state, accepting,initial	] (1) {$1$};
% 			\node[state, right of=1			] (3) {$3$};
% 			\draw	(1)	edge[bend right, below	] node{$(a+b)a^*b$} (3)
% 					(3)	edge[bend right, above	] node{$a$} (1)
% 					(3) edge[loop above			] node{$ba^*b$} (3);
% 		\end{tikzpicture} &
% 		\begin{tikzpicture}[->,>=stealth',node distance=2.5cm,initial text=$ $,]
% 			\node[state, initial	] (1) {$1$};
% 			\node[state, accepting, right of=1	] (3) {$3$};
% 			\draw	(1)	edge[bend right, below	] node{$(a+b)a^*b$} (3)
% 					(3)	edge[bend right, above	] node{$a$} (1)
% 					(3) edge[loop above			] node{$ba^*b$} (3);
% 		\end{tikzpicture}
% 		\\
% 		\begin{tikzpicture}[->,>=stealth',node distance=2.5cm,initial text=$ $,]
% 			\node[state, accepting,initial	] (1) {$1$};
% 			\draw	(1) edge[loop above			] node{$(a+b)a^*b(ba^*b)^*a$} (1);
% 		\end{tikzpicture} & \\
% 		$((a+b)a^*b(ba^*b)^*a)^*$ & $((a+b)a^*b(ba^*b)^*a)^*(a+b)a^*b$
% 	\end{tabular}
% \end{center}
% \begin{alignat*}{12}
% 	((a+b)a^*b(ba^*b)^*a)^* + ((a+b)a^*b(ba^*)^*a)^*(a+b)a^*b
% 	&= ((a+b)a^*b(ba^*b)^*a)^*(\varepsilon+(a+b)a^*b)
% \end{alignat*}
% \pagebreak
% \questionitem{Item e}
% \begin{center}
% 	\begin{tabular}{c c}
% 		\begin{tabular}{r | c c}
% 			$\delta_N$        & $a$           & $b$ \\ \hline
% 			$\rightarrow q_0$ & $\{q_0,q_1\}$ & $\emptyset$ \\
% 			$         ^* q_1$ & $\emptyset$ & $\{q_0\}$
% 		\end{tabular} &
% 		\begin{tabular}{r | c c}
% 			$\delta_D$            & $a$           & $b$ \\ \hline
% 			$          \emptyset    $ & $\emptyset  $ & $\emptyset$ \\
% 			$\rightarrow \{q_0    \}$ & $\{q_0,q_1\}$ & $\emptyset$ \\
% 			$         ^* \{q_0,q_1\}$ & $\{q_0,q_1\}$ & $\{q_0\}$
% 		\end{tabular}
% 	\end{tabular}
% \end{center}
% \questionitem{Item f}
% \begin{minipage}[c]{0.58\textwidth}
% 	$\{q_0,q_1\}$ is distinguishable from $\emptyset$ and $\{q_0\}$, given the first state is final, and the last two states are not final.\par
% 	Also, $\{q_0\}$ and $\emptyset$ are distinguishable, because for them to be indistinguishable it was required that $\emptyset$ and $\{q_0,q_1\}$ were undistinguishable (which is not true).
% 	\begin{center}
% 		\begin{tabular}{r || p{15mm} | p{15mm} | p{15mm}}
% 							& $\emptyset$ & $\{q_0\}$ & $^* \{q_0,q_1\}$ \\ \hline \hline
% 			$\emptyset$      & \cellcolor{gray}  & \cellcolor{gray}  &\cellcolor{gray} \\ \hline
% 			$\rightarrow \{q_0\}$        & X & \cellcolor{gray}  &\cellcolor{gray} \\ \hline
% 			$^* \{q_0,q_1\}$ & X & X & \cellcolor{gray}
% 		\end{tabular}
% 	\end{center}
% \end{minipage}
% \begin{minipage}[c]{0.4\textwidth}
% 	\begin{center}
% 		\begin{tikzpicture}[->,>=stealth',node distance=2.5cm,initial text=$ $,]
% 			\node[state, initial					] (q0) {$q_0$};
% 			\node[state, below right of=q0					] (e) {$\emptyset$};
% 			\node[state, accepting,above right of=e,align=center	] (q0_q1) {$q_0$\\$q_1$};
			

% 			\draw	(e)  edge[loop below	] node{$a,b$} (e)
% 					(q0) edge[bend left, above			] node{$a$} (q0_q1)
% 					(q0) edge[left			] node{$b$} (e)
% 					(q0_q1) edge[loop above] node{$a$} (q0_q1)
% 					(q0_q1) edge[bend left, above] node{$b$} (q0);
% 		\end{tikzpicture}
% 	\end{center}
% \end{minipage}
% \examgroup{Group II}
% \questionitem{Item a}
% \begin{theorem}
% 	Given $L_1=\{a^n b^n \mid n \geq 0\}$ is a non-regular language, $L_2=\{0^n 1^m 2^{n-m} \mid  n \geq m \geq 0\}$ is also a non-regular language.
% \end{theorem}
% \begin{proof}
% 	Assume by absurd that $L_2$ is a regular language. We know that regular languages are closed to homomorphism. Thus, by defininf $L_1=h(L_2)$ where:
% 	\begin{alignat*}{2}
% 		h \colon \{0,1,2\} &\rightarrow \{a,b\}\\
% 		0                  &\mapsto a\\
% 		1                  &\mapsto b\\
% 		2                  &\mapsto b
% 	\end{alignat*}
% 	we are implying that $L_1$ is also a regular language. We thus arrive at a contradiction with the statement of the theorem, thus proving the theorem correct.
% \end{proof}
% \questionitem{Item b}
% Regular languages are closed under the symmetric difference operator, given there is an algorithm to find the regular expression of the symmetric difference of two languages given by regular expressions.\par
% Let $L_1$, $L_2$ be the regular languages for which we have regular expressions.
% \begin{enumerate}
% 	\item Convert the two regular expressions to DFAs, $D_1$ and $D_2$.
% 	\item Obtain $D=D_1 \times D_2$, the cartesian product of $D_1$ and $D_2$, where a state $(q_1,q_2)\in Q_D$ (where $q_1 \in Q_{D_1}$ and $q_2 \in Q_{D_2}$) is final if exactly one of $q_1$ or $q_2$ are final states of their respective original DFAs. Thus, we are accepting strings that are accepted either by $L_1$ or by $L_2$, but not by both.
% 	\item Convert $D$ to a regular expression.
% \end{enumerate}
% We thus arrive at a regular expression of the symmetric difference of two languages.
% \pagebreak
% \examgroup{Group III}
% \begin{alignat*}{2}
% 	L=\{a^n w w^R a^j \mid n \geq 0, n \leq j \leq 2n, w \in \{a,b\}^*\}
% \end{alignat*}
% \questionitem{Item a}
% \begin{alignat*}{2}
% 	S &\rightarrow aSa\mid aSaa\mid W\\
% 	W &\rightarrow aWa\mid bWb\mid \varepsilon
% \end{alignat*}
% \questionitem{Item b}
% \begin{minipage}[c]{0.68\textwidth}
% 	\begin{alignat*}{12}
% 		S \implies aSa \implies aaSaa \implies aaWaa \implies aabWbaa \implies aabbaa
% 	\end{alignat*}
% \end{minipage}
% \begin{minipage}[c]{0.3\textwidth}
% 	\begin{center}
% 		\begin{tikzpicture}
% 			\Tree	[.S
% 						a
% 						[.S
% 							a
% 							[.S
% 								[.W
% 									b
% 									[.W $\varepsilon$ ]
% 									b
% 								]
% 							]
% 							a
% 						]
% 						a
% 					]
% 		\end{tikzpicture}
% 	\end{center}
% \end{minipage}
% \questionitem{Item c}
% Yes it is ambiguous, given string $aabbaa$ has at least two different syntax trees: the one presented in the previous item, and the following syntax tree.
% \begin{center}
% 	\begin{tikzpicture}
% 		\Tree	[.S
% 					a
% 					[.S
% 						[.W
% 							a
% 							[.W
% 								b
% 								[.W $\varepsilon$ ]
% 								b
% 							]
% 							a
% 						]
% 					]
% 					a
% 				]
% 	\end{tikzpicture}
% \end{center}
% An equivalent, non-ambiguous grammar is:
% \begin{alignat*}{2}
% 	S &\rightarrow aSaa\mid W\\
% 	W &\rightarrow aWa\mid bWb\mid \varepsilon
% \end{alignat*}
% \questionitem{Item d}
% \begin{minipage}[c]{0.49\textwidth}
% 	\begin{equation*}
% 		PDA~P=(\{q\},\{a,b\},\{a,b,S,T,W,V\},\delta,q,S)
% 	\end{equation*}
% 	\begin{alignat*}{2}
% 		\delta(q,\varepsilon,S)&=\{(q,aSaa),(q,W)\} \\
% 		\delta(q,\varepsilon,W)&=\{(q,aWa),(q,bWb),(q,\varepsilon)\}\\
% 		\delta(q,a,a)&=\{(q,\varepsilon)\}\\
% 		\delta(q,b,b)&=\{(q,\varepsilon)\}\\
% 	\end{alignat*}
% \end{minipage}
% \begin{minipage}[c]{0.49\textwidth}
% 	\begin{center}
% 		\begin{tikzpicture}[->,>=stealth',node distance=2.5cm,initial text=$ $,]
% 			\node[state, initial						] (q) {$q$};
% 			\draw	(q)  edge[loop above,align=center	] node{
% 				$\varepsilon,S/aSaa$\\
% 				$\varepsilon,S/W$ \\
% 				$\varepsilon,W/aWa$\\
% 				$\varepsilon,W/bWb$\\
% 				$\varepsilon,W/\varepsilon$\\
% 				$a,a/\varepsilon$\\
% 				$b,b/\varepsilon$
% 			} (q);
% 		\end{tikzpicture}
% 	\end{center}
% \end{minipage}
% \questionitem{Item e}
% \begin{alignat*}{5}
% 	(q,aabbaa,S) &\vdash (q,aabbaa,W) &&\vdash (q,aabbaa,aWa) &&\vdash (q,abbaa,Wa) &&\vdash (q,abbaa,aWaa) \\
% 				 & \vdash (q,bbaa,Waa) &&\vdash (q,bbaa,bWbaa) &&\vdash (q,baa,Wbaa)&&\vdash (q,baa,baa) \\
% 				 & \vdash (q,aa,aa)     &&\vdash (q,a,a) &&\vdash (q,\varepsilon,\varepsilon)
% \end{alignat*}
% \examgroup{Group IV}
% \questionitem{Item a}
% A strategy to implement a TM to answer this question would be:
% \begin{enumerate}
% 	\item Find first 0/1, mark it with $X$ and go to corresponding state if found a 0 or a 1.
% 	\item Find first 1/0. If the corresponding pair was found, mark the new symbol with $X$, go to beginning of string and go to step (1). If the head of the TM reaches the end of the string after having found a 1 but not a zero, the TM accepts the input.
% \end{enumerate}
% \questionitem{Item b}
% \begin{minipage}[c]{0.56\textwidth}
% 	\begin{center}
% 		\begin{tabular}{r | c c c c}
% 								& 0     				& 1 					& $X$ 					& $B$ \\ \hline
% 			$\rightarrow s  $	& $(g_0,X,\rightarrow)$ & $(g_1,X,\rightarrow)$ & $(s  ,X,\rightarrow)$ &                       \\
% 			$            g_0$   & $(g_0,0,\rightarrow)$ & $(l  ,X,\leftarrow )$ & $(g_0,X,\rightarrow)$ &                       \\
% 			$            g_1$	& $(l  ,X,\leftarrow )$ & $(g_1,1,\rightarrow)$ & $(g_1,X,\rightarrow)$ & $(f  ,B,\rightarrow)$ \\
% 			$            l  $	& $(l  ,0,\leftarrow )$ & $(l  ,1,\leftarrow )$ & $(l  ,X,\leftarrow )$ & $(s  ,B,\rightarrow)$ \\
% 			$         ^* f  $   &                       &                       &                       &
% 		\end{tabular}
% 	\end{center}
% \end{minipage}
% \begin{minipage}[c]{0.43\textwidth}
% 	\begin{center}
% 		\begin{tikzpicture}[->,>=stealth',node distance=3.2cm,initial text=$ $,]
% 			\node[state, initial			] (s) {$s$};
% 			\node[state, above right of=s	] (g0) {$g_0$};
% 			\node[state, below right of=s	] (g1) {$g_1$};
% 			\node[state, below right of=g0	] (l) {$l$};
% 			\node[state, below right of=g1	] (f) {$f$};
			
% 			\draw	(s)  edge[left			] node{$0/X\rightarrow$} (g0)
% 					(s)  edge[right			] node{$1/X\rightarrow$} (g1)
% 					(s)  edge[loop below	] node{$X/X\rightarrow$} (s)

% 					(g0)  edge[loop above,align=center	] node{$0/0\rightarrow$\\$X/X\rightarrow$} (g0)
% 					(g0)  edge[left					] node{$1/X\leftarrow$} (l)

% 					(g1)  edge[loop below,align=center	] node{$1/1\rightarrow$\\$X/X\rightarrow$} (g1)
% 					(g1)  edge[right					] node{$0/X\leftarrow$} (l)
% 					(g1)  edge[right					] node{$B/B\rightarrow$} (f)
					
% 					(l)  edge[loop above,align=center	] node{$0/0\leftarrow$\\$1/1\leftarrow$\\$X/X\leftarrow$} (l)
% 					(l)  edge[above						] node{$B/B\rightarrow$} (s)
% 					;
% 		\end{tikzpicture}
% 	\end{center}
% \end{minipage}
% \questionitem{Item c}
% \begin{alignat*}{12}
% 			& B s 101B &&\vdash BX g_1 01B &&\vdash B l XX1B &&\vdash l BXX1B &&\\
% 	\vdash 	& B s XX1B &&\vdash BX s X1B &&\vdash BXX s 1B &&\vdash BXXX g_1 B &&\vdash BXXXB f 
% \end{alignat*}
% \examgroup{Group V}
% \begin{center}
% 	\begin{tabular}{c | c p{132mm}}
% 		\textbf{(a)} & True & All finite languages are regular. All regular languages are representable as regular expressions, where a regular expresison is nothing more than the result of applying certain regular operations over finite languages. \\ \hline
% 		\textbf{(b)} & True & A grammar representing that language is $S \rightarrow aaSbbb \mid  \varepsilon$. \\ \hline
% 		\textbf{(c)} & True & 
% 		\begin{minipage}[c]{0.6\textwidth} \vspace*{0.3em}
% 			The CFG is equivalent to the following DFA:\\
% 			\begin{tikzpicture}[->,>=stealth',node distance=2.5cm,initial text=$ $,]
% 				\node[state, initial		] (q0) {$q_0$};
% 				\node[state, right of=q0	] (z) {$z$};
% 				\node[state, below right of=q0] (f) {$f$};
				
% 				\draw	(q0)  edge[bend left=7, above	] node{$0$} (z)
% 						(q0)	edge[loop below			] node{$1,2$} (q0)
% 						(z)  edge[bend left=7, below	] node{$0$} (q0)
% 						(q0) edge[right					] node{$3$} (f)
% 						;
% 			\end{tikzpicture}  \vspace*{0.3em}
% 		\end{minipage} \\ \hline
% 		\textbf{(d)} & False & Say $L$ is not a CFL. $\Sigma^*$ is a regular language (with regular expression $(a+b)^*$), and thus also a CFL. $L \subset \Sigma^*$, but $L$ is not a CFL and $\Sigma^*$ is a CFL. \\ \hline
% 		\textbf{(e)} & True & $L \text{ non-regular} \rightarrow L^C \text{ non-regular} \iff L^C \text{ regular} \rightarrow  (L^C)^C \text{ regular} \iff M \text{ regular} \rightarrow  M^C \text{ regular}$, which is trivially true given regular languages are closed to complement (in a DFA, just swap accepting states for normal and vice-versa).\\ \hline
% 		\textbf{(f)} & False & Say $L$ is not regular. By the previous item, $L^C$ is not regular either. However, $L \cup L^C = \Sigma^*$, and $\Sigma^*$ is a regular language.\\ \hline
% 		\textbf{(g)} & True & A PDA with $M$ states and finite stack of size $N$ can be represented by an $\varepsilon$-NFA with less than $M(\#\Gamma)^N$ states: for each of the $M$ states, we can enumerate all possible stack configurations (up to $(\#\Gamma)^N$) and know the next state and stack configuration. \\ \hline
% 		\textbf{(h)} & True & A Turing machine is more powerful than a DFA. Also, consider the DFA $D$ equivalent to the regular expression. Now transform it into a Turing machine, using the following rules:
% 		\begin{itemize}
% 			\itemsep0em
% 			\item All states and transitions are the same in the TM as in the DFA.
% 			\item The TM only moves the head to the right.
% 			\item All accept states $q \in Q_D$ are transformed into TM transitions ${\delta(q,B)=(f,B,\rightarrow)}$ where $f$ is the single final state of the TM.
% 		\end{itemize} \\ \hline
% 		\textbf{(i)} & True & A PDA with two stacks is equivalent in power to a TM (meaning a normal PDA with one stack has less power than a TM), and there are algorithms that convert a PDA to a Turing Machine, where one only needs to make some considerations on how to implement the stack and manage pushes/pops.
% 	\end{tabular}
% \end{center}
% }
\end{document}
